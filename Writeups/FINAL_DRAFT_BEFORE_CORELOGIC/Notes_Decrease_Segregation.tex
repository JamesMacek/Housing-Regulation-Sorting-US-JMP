\documentclass[]{article}
\usepackage{setspace}
\usepackage{amsmath}

\doublespacing

%opening
\title{Why can exclusionary zoning decrease segregation?}
\author{James Macek}

\begin{document}

\maketitle

\section*{}
\paragraph*{}
The output of my model comes with a surprising result: income segregation can \textit{increase} after eliminating regulation that, \textit{in isolation}, can cause segregation on income. Central to this result is the pattern of \textit{unobserved} neighborhood characteristics that cause income sorting. It turns out that these unobserved characteristics are \textit{negatively correlated} with the stringency of lot size regulation. The regulation is "correcting" any segregation that would have taken place in its absence. 
\paragraph*{}
To make this clear, I want to review what happens in the counterfactual deregulation exercise to both high density and low density neighborhoods of expensive cities relative to cheap ones. Recall that there is generally stronger negative income sorting on density in expensive cities (Fact 2 in the paper). This means that high density neighborhoods tend to be relatively less affluent. However, after deregulation, this pattern reverses completely. That is, high density neighborhoods (near central cities) become relatively more affluent in expensive metros; something that looks a lot like gentrification. These high density neighborhoods tend to have low levels of lot size regulation. This means that unobserved factors in these neighborhoods would cause positive income sorting, \textit{but for regulation in the low density neighborhoods}.  The emergence of affluent central cities are the reason why the deregulated equilibrium is more segregated!

\paragraph*{}
A model can solidify this logic. It will abstract away from an explicit model of minimum lot sizes (I have such a model in my paper) and instead consider regulation as an amenity that \textit{disproportionately benefits high income workers}. This is only to illustrate the mechanism. 
\paragraph*{}
Consider a closed city with two locations, $r_{1}$ and $r_{2}$ and index an arbitrary location by $r$. The implicit ordering of these locations will correspond to income sorting patterns; how will be made clear in a moment. There are two types of workers, each earning income $z_{1}$ and $z_{2}$ with $z_{1} < z_{2}$. Assume a total mass of workers of $1$ for each type, and let $z$ index the income of a worker of arbitrary type. Each worker $z$ faces a set of housing prices $\{P(r)\}_{r \in \{r_{1}, r_{2}\}}$, and chooses housing $A$ and numeraire consumption $c$ to solve

\begin{equation}
	\max_{c, A} A^{\beta}c^{1-\beta}
\end{equation}
subject to $P(r)A + c \leq z$. Let $V(r, z)$ be the solution to this problem. 

\paragraph*{}
 Each worker enjoys a type specific amenity $b(r, z)$ for residing in neighborhood $r$. This amenity enters multiplicatively, so that the utility from neighborhood $r$ is $V(r, z)b(r, z)$. Because of the Cobb-Douglas structure of preferences, relative differences in these amenities across workers will be the sole reason for income sorting. In a simple Frechet model of migration, one can show that the affluence of location $r_{1}$ relative to $r_{2}$ can be expressed as
 
 \begin{equation}
 	\frac{L(r_{1}, z_{2})}{L(r_{1}, z_{1})}/\frac{L(r_{2}, z_{2})}{L(r_{2}, z_{1})} = \bigg[\frac{b(r_{1}, z_{2})}{b(r_{1}, z_{1})}/\frac{b(r_{2}, z_{2})}{b(r_{2}, z_{1})}\bigg]^{\theta}
 \end{equation}
 where $L(r, z)$ is the equilibrium population of a type $z$ worker in location $r$, and $\theta > 0$ is an arbitrary migration elasticity with respect to real income. Equation (2) holds in any parameterization of the model, and for any realization of housing prices.  It makes sense to consider the following measure of segregation $S$ derived from Equation (2)
 \begin{equation}
 	S := \left|\left|\log\bigg[\frac{L(r_{1}, z_{2})}{L(r_{1}, z_{1})}/\frac{L(r_{2}, z_{2})}{L(r_{2}, z_{1})}\bigg]\right|\right|
 \end{equation}

where $\left|\left|.\right|\right|$ is the absolute value operator. If this measure large, then that means there are relatively more high income people in some location relative to another.

\paragraph*{}
Consider a scenario where location 2 is relatively affluent because of lot size regulation, for example,

\begin{equation*}
\frac{b(r_{1}, z_{2})}{b(r_{1}, z_{1})}/\frac{b(r_{2}, z_{2})}{b(r_{2}, z_{1})}	= 2.
\end{equation*}
However, after deregulation, the relative amenity value location $2$ falls to 

\begin{equation*}
	\frac{b(r_{1}, z_{2})}{b(r_{1}, z_{1})}/\frac{b(r_{2}, z_{2})}{b(r_{2}, z_{1})} = \frac{1}{3}
\end{equation*}
\paragraph*{}
In the deregulated equilibrium, segregation $S$ has increased from $\theta\log(2)$ to $\theta\log(3)$ using Equation (2) and (3). 

\paragraph*{}
Think of the location $r_{2}$ as a suburb that is nice for affluent households only because of large lots. Think of location $r_{1}$ as an unregulated central city that has amenities valued disproportionately by the rich, but has small housing units that the rich don't like. \textit{But for the regulation}, $r_{1}$ would be relatively more affluent. This echos the same message in the second paragraph of this note. 

\end{document}
