\documentclass[11pt]{article}
\usepackage{amsmath}
\usepackage{amssymb}
\usepackage{hyperref}
\usepackage{palatino}
\usepackage{graphicx}
\usepackage{setspace}
\usepackage{amsthm, thmtools}
\usepackage[legalpaper, margin=1in]{geometry}

%Propositions, facts, remarks
\declaretheorem{Fact}
\declaretheorem{Proposition}
\declaretheorem{Remark}


\usepackage[font=scriptsize,labelfont=bf]{caption}
\bibliographystyle{chicago}
\spacing{1.3}
\hypersetup{
	colorlinks=true,
	linkcolor=blue,
	filecolor=blue,      
	urlcolor=blue,
	citecolor=blue,
}
\usepackage{comment}

\usepackage{natbib}
\bibliographystyle{chicago}

%opening
\title{Optimal Policy}
\author{James Macek}
\date{March 2024}

\begin{document}
	
	\maketitle

	
	\section{Optimal Policy Definition}
	\paragraph*{}
	Fix some level of regulation $R(i)$ and consider equilibrium outcomes $P_{o}(i)$, $L_{o}(i, z)$, solving the equilibrium conditions. The \textbf{Social planner's problem} for a set of weights $\alpha(z)$ is defined as 
	
	\begin{equation*}
		\max_{}
	\end{equation*}
	
	
	
	\section{The Marginal Returns to Deregulation}
	
	\section{Section: does the no regulation equilibrium with no externalities solve an SPP for some weights?}
	
	\section{}
	
	\paragraph*{}
	
	
	
	
	
		
\end{document}