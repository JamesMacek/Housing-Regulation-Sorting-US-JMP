\documentclass[]{article}
\usepackage{amsmath}
\usepackage{hyperref}
\usepackage{palatino}
\usepackage{setspace}
\usepackage[font=small,labelfont=bf]{caption}
\bibliographystyle{chicago}
\onehalfspacing
\hypersetup{
	colorlinks=true,
	linkcolor=blue,
	filecolor=blue,      
	urlcolor=blue,
	citecolor=blue,
}


\usepackage{natbib}
\bibliographystyle{chicago}


%opening
\title{Lot Sizes, Welfare and Urban Form: A View from the United States \\
Research Proposal}
\author{James Macek}

\begin{document}

\maketitle

\begin{abstract} 
\end{abstract}

\section{Introduction}
\paragraph*{}
In recent decades, high housing prices in certain "superstar" cities have been the center of attention in urban policy \citep{superstarcities}. These rising prices are met with unequal consequences for those priced out of desirable labour markets, or pay large fractions of income on rents \citep{nonhomo}, with broader implications for the economy at large \citep{hseihmoretti} \citep{parkho}. A growing set of research seeks to measure how land use regulations contribute to these high prices. In this paper, I focus on measuring the welfare impacts of a particular set -- minimum lot sizes and unit density restrictions. These restrictions are ubiquitous and have a growing prevalence in the United States \citep{gyourko2021}.  
\paragraph*{}

 In my model, these regulations keep housing prices high by increasing the supply of housing stock per housing unit, pushing low income  households to other neighborhoods or cities that are cheaper or allow denser housing. This channel alters housing prices and wages in those locations, in addition to local amenities and the provision of public goods; shaping urban form both \textit{within} and \textit{across} cities. This motivates special attention to how these welfare estimates are shaped by general equilibrium effects at both a small and large spatial scale.
\paragraph*{}
Why are minimum lot sizes important, and what do I do differently to cost them? Firstly, there is growing evidence that these regulations cause sorting on income \citep{kulka} \citep{Song}. In a similar vein, I establish that there is negative income sorting into high density neighborhoods within cities, and that these sorting patterns are generally stronger in expensive ones. This has broader implications when one accounts for positive reinforcement of these sorting patterns. Relevant channels include endogenous amenities\footnote{See, for example, \cite{Coutureetal}, \cite{AlmagroDI}, \cite{diamond2016} and \cite{bshartley2020}. This literature also emphasizes heterogenous valuations of these amenities across groups, and similar effects appear in my model.}, as well as the provision of congested public goods in the zoning context \citep{calabresetal} \citep{ineffTiebout}. Unfortunately, these demand side effects have received little attention in computing the aggregate welfare impacts of regulations. To see why this matters, consider a thought experiment surrounding the mechanisms in \cite{hseihmoretti} and \cite{durantonpugaurbgrowth}, among others. The standard story is that regulations slow aggregate growth by preventing workers from accessing productive local technologies that are responsible for this growth. However, loosening restrictions in expensive cities in the presence of income sorting causes high income, productive households to leave, attenuating these negative effects. Even so, this does not factor in the externalities implied by these sorting mechanisms\footnote{For example, exclusionary zoning (and, as I show, lot size restrictions) can be optimal to limit the fiscal deficit of high income residents in the presence of property taxes and Tiebout sorting. This is the classic argument made in \cite{hamilton1976}. Some models predict inefficiently stringent levels of exclusionary zoning under reasonable calibrations, e.g.  \cite{ineffTiebout}.}. My first contribution is to square these competing ideas in a general equilibrium framework, with the added benefit of considering the trade off between equity and efficiency.     

\paragraph*{}
Secondly, I provide observational evidence that heterogeneous lot size and unit density restrictions have altered urban form in expensive cities. In particular, they appear to suppress the relative number of housing units in medium density neighborhoods when comparing expensive cities to cheap ones. Likewise, high density locations near downtowns and low density locations on the urban fringe appear relatively larger\footnote{This idea might be surprising because it differs from the effects of other closely related regulations on urban form. For example, \cite{bbheight} and \cite{mills2005} show that height or housing stock density restrictions flatten the urban density gradient, if said height restrictions are uniform across space. This cannot explain the result downtown.}. Accounting for within-city geography is crucial for gauging the welfare impacts of relaxing these regulations for three reasons. Firstly, households can move \textit{within} cities to avoid stringent lot sizes if there is significant within-city heterogeneity in them. Secondly, lot sizes may not be stringent if they tend to be located far from employment centers. The evidence I provide suggest both mechanisms are at play. This \textit{urban form} channel is generally ignored in the macro literature through spatial aggregation, but enters into my model\footnote{A notable exception is \cite{durantonpugaurbgrowth}, and I elaborate on how I differ when reviewing the literature. This low level of spatial aggregation is also useful to allow flexibility in modeling the spatial spillovers that cause neighborhood income sorting. For example, I can construct a model in which endogenous amenities operate at the school district level, in the spirit of the zoning and public goods literature. I can compare and contrast that with a model where spillovers operate at the tract level  (are more local).  This may be ambitious, and I will move to a higher level of aggregation if necessary.}.

\paragraph*{}
Studying housing regulation is challenging because it is difficult to measure, especially with broad geographic coverage. I adapt the procedure to detect minimum lot sizes used in \cite{Song}, leveraging assessment data. I possibly may generalize this methodology to height restrictions, because these are important regulations that complement lot sizes \citep{KSC}. I sketch this procedure in the body of this note.
\paragraph*{}
These observed minimum lot sizes enter directly into calibration of the model. With them, I construct a model-based measures of minimum housing quality per housing unit that arise from the developer's problem, leveraging micro-geographic variation in housing supply elasticities from \cite{BSH}. These cause income sorting into neighborhoods in the model. In addition, I allow for income per capita to endogenously increase neighborhood amenities. Estimation of this relationship runs into an identification problem similar to \cite{diamond2016}, and others. I instrument for neighborhood income per capita with within-city variation in natural terrain slopes, controlling for other natural amenities that would presumably cause income sorting. This instrument is highly relevant; I elaborate in detail at the end of this note.

\section{Literature}
\paragraph*{}
Given the spatial scale of the model, I see this paper as firstly improving upon the literature that quantifies the macroeconomic impacts of housing regulation. These papers generally employ variations of the Rosen-Roback framework, consider homogenous households, and derive welfare implications from calibrated models. The two most relevant to my work are \cite{hseihmoretti} and \cite{durantonpugaurbgrowth}. Firstly, \cite{hseihmoretti} show that growth in the US was slowed because this growth was concentrated in cities with inelastic housing supply. Importantly, they also measure the welfare impacts of regulation through their effect on housing supply elasticities, using estimates from \cite{saiz2010}. On the other hand, \cite{durantonpugaurbgrowth} consider a model (FINISH). 
\paragraph*{}
 \cite*{hop} use model-based residuals to measure land use regulation over time, and perform similar counterfactuals. \cite{ganongshoag} link housing regulation to declining spatial convergence in US income. \cite{parkho} and \cite{bunten} construct aggregate welfare estimates of deregulation in settings where they are determined in a political economy equilibrium, and find attenuated gains\footnote{These estimates are also not directly comparable to \cite{hseihmoretti} because they are built under models with local increasing returns to scale. That is, the misallocation mechanism of \cite{hseihmoretti} is not present in them.}.  The former also considers national policies and how local regulation might respond to them. On a smaller spatial scale, \cite{martynov} and \cite{acosta} measure the impacts of housing stock density restrictions where commuting plays a first order role. 

\paragraph*{}
I build on all these papers by focusing on the income sorting mechanism both within and across cities, and consider that household movement within those cities to avoid stringent lot size regulation. In doing so, I draw ideas from other literatures under the umbrella of urban and public economics. I review them in sequence:

\paragraph*{Housing regulation} This paper adds to the historic and fast growing literature that studies housing regulation, as recently reviewed in \cite{gyourkomolloy}. In the context of minimum lot sizes, recent quasi-experimental evidence from \cite{Song} and \cite{kulka} show extreme sorting on income and race into stringent municipalities. \cite{KSC} and \cite{zabel} find significant effects of similar regulations on local prices. These papers provide a strong motivation for thinking about these channels in a general equilibrium context. Importantly, the former papers' models do not incorporate endogenous amenities\footnote{\cite{Song} directly models preferences over minimum lot sizes in a discrete choice model. This may be thought of as a stand-in for endogenous amenities, though this remains unclear.}, nor do they model heterogeneous local technologies that are important for assessing the aggregate welfare consequences of deregulation.
\paragraph*{}
Beyond these, the literature on lot sizes and housing unit density restrictions is surprisingly scant relative to research on housing \textit{stock} density restrictions. Theoretical treatments include \cite{bbheight} and \cite{mills2005}, who show these restrictions cause urban sprawl. On the empirical side, \cite{BruecknerFuGu} and \cite{bruecknersingh} use a simple model to measure the stringency of FAR restrictions in China and the US, and \cite{TAN2020} generalize this procedure to a setting where stock differs on latent quality. These restrictions are sometimes thought to be isomorphic to minimum lot sizes, but I show that they are not --  a distinction made in an old literature \citep{griesonwhite}.


\paragraph*{Urban spatial sorting} A growing literature in urban economics is seeking to understand the causes and consequences of the spatial sorting of different groups, both within and across cities. Usually, these papers stress that sorting in response to changing exogenous fundamentals are exacerbated by endogenous forces, such as changing amenities. \cite{diamond2016} show that the rapidly evolving skill intensity in large cities was largely driven through amenities valued by the high skilled. \cite{couturehandbury} and \cite{su2021} show that non-traded amenities and the rising value of time have contributed to the changing skill intensity in downtowns, respectively. \cite{bshartley2020} show a similar result for recently observed racial sorting. \cite{Coutureetal} show that the evolving national income distribution has contributed to the recent gentrification in US downtowns.

\paragraph*{}
Unlike these papers, regulated lot sizes are the anchor that cause neighborhood income sorting in my model. However, I want to stress that it cannot explain these recent gentrification patterns, and instead speaks to an older literature on urban decay; including one that explains negative income sorting into downtowns \citep{parispoor} \citep{ccpoortransport}. Moreover, my model cannot explain how income sorting is shaped by durable housing and the spatiotemporal patterns of filtering \citep{Gentrificationcycles}. 

\paragraph*{}
There may also be a link with the urban inequality literature, though I have yet to flush that out in a model \citep{ineqcitysize}, \citep{spatialsorting}, \citep{FogliGuerrieri}. Within-city heterogeneity in lot sizes may play a role.

\paragraph*{Exclusionary zoning}
This paper is also closely linked to a literature which studies the role that zoning policy plays in the presence of Tiebout sorting and ad valorem property taxes. In general, these papers consider models where households vote over the minimum amount of housing services households must purchase to live in a jurisdiction. In equilibrium, rich households exclude the poor to bolster their fiscal surplus with respect to a local public good. \cite{calabresetal} and \cite{keepingpeopleout} consider a model where zoning policy is collectively chosen before and after neighborhood choice, respectively. They both generally find regressive welfare effects, but find substantial efficiency gains. In contrast, \cite{ineffTiebout} calibrate a model where fiscal decentralization fails to materialize into efficiency gains. A similar conclusion is reached by \cite{barcoate} in a dynamic framework. In my model, the externality that gives rise to urban spatial sorting can be interpreted in the context of these papers. However, I do not endogenize the choice of local zoning policy, and instead consider it as a lever than can be freely changed. 


\section{Motivating Evidence}
Just adapt stuff I had from the Missing Middle presentation (income sorting + urban form). Update it to use the land assessment data provided by Nate (to measure the fraction of land used by single family homes). Will be done for the presentation.

\section{Model}
\paragraph*{Geography}
I consider a finite set of cities $C$ indexed by $c$, which map to MSAs in the data. These cities are self-contained labour markets; that is, I do not allow households to access local technologies outside of the city for which they reside. Each city $c$ has an exogenous finite set of neighborhoods $N(c)$. I use the index $i$ to denote a typical neighborhood from any city, $i \in \cup_{c \in C}N(c)$, and define the map $C(i)$ to be the city associated with $i$. Here, neighborhoods may be census tracts, or some aggregation thereof, depending on computational issues. 
\paragraph*{}
Each neighborhood has an exogenous amount of land $T_{R}(i)$ that can be devoted to residential use, and each city has an exogenous amount of land $T_{C}(c)$ that can be used for commerical purposes at a central business district. I adopt a monocentric interpretation of the city for good reason, which I will elaborate on. Unfortunately, this requires that land be immobile across uses, else it would demand that households can live and work in any neighborhood\footnote{\cite{acosta} and \cite{martynov} consider a model of housing stock density regulation featuring arbitrary workplace and residence locations.}. 

\paragraph*{Developer's Problem}   
Complicating an otherwise standard specification of housing supply is the minimum lot size $\bar{l}(i)$ and the regulation of how many housing units can occupy each lot. To make the exposition clear, I start with preliminaries. In each neighborhood, I partition residential land $T_{R}(i)$ into equal sized \textit{parcels} of normalized mass $1$. That is, there are a mass $T_{R}(i)$ of parcels. These are the units of land that the representative developer uses to produce structure. These parcels can be \textit{split} into lots, and lots are a fundamental unit by which regulation operates. Each lot can hold a regulated maximum $\bar{h}(i)$ housing units. Only one household can occupy a housing unit. For example, $i$ may allow duplexes, so that $\bar{h}(i) = 2$. Then, $\bar{l}(i)/\bar{h}(i)$ is the minimum amount of land per housing unit in $i$. Let $l(i)$ denote this minimum land per housing unit, which will be the main object I work with hereafter\footnote{I'm making a very implicit assumption here, i.e. there is no material difference between single family homes on small lots or multifamily homes on large lots.}. Of course, $l(i)$ may be zero if the neighborhood is unregulated. 
\paragraph*{}
Given a parcel in $i$, developers choose the total amount of structure $A(i)$ that can occupy it in a standard way. That is, they use a neighborhood-varying Cobb-Douglas technology over land and capital, facing a perfectly elastic supply of that capital at rate $r$. This yields the neighborhood-level housing supply function
\begin{equation}\label{supplyfn}
	A(i) = \bigg(\frac{P(i)}{\lambda(i)r}\bigg)^{\epsilon(i)}T_{R}(i)
\end{equation}
where $P(i)$ is the price of an effective unit of housing stock, $\epsilon(i)$ is the supply elasticity in tract $i$ and $\lambda(i)$ is a supply shifter. In a world without minimum lot sizes, the developer is indifferent to allocating this structure across housing units; there can be many small houses or few large ones, provided the total stock is given by \eqref{supplyfn}. Instead, if developers respect the minimum lot size, the minimum amount of housing stock per housing unit $A^{\star}(i)$ must be

\begin{equation}\label{minstructure}
	A^{\star}(i) = \bigg(\frac{P(i)}{\lambda(i)r}\bigg)^{\epsilon(i)}l(i)
\end{equation}

\paragraph*{}
Equation \eqref{minstructure} reveals the material difference between lot size regulation and other regulations studied in contemporary quantitative models. Contrast the equation with the standard Floor Area Ratio restriction in \cite{bruecknersingh} or \cite{BruecknerFuGu}, which puts limits on housing \textit{stock} per parcel. Here, there are no stock density  limits -- just limits on the number of housing units that can occupy a parcel. This distinction is forcefully argued in \cite{griesonwhite}.

\paragraph*{}
Equation \eqref{minstructure} also reveals how minimum lot sizes and housing unit density restrictions cause income sorting. I take the quantity $A^{\star}(i)$ as a minimum amount of housing stock required to be purchased to live in neighborhood $i$; households with income below $P(i)A^{\star}(i)$ will be priced out of the local housing market. This quantity is increasing in housing prices, holding lot sizes fixed. Moreover, households with little disposable income after paying for this minimum quantity are forced to purchase more than what they would if this quantity could be freely chosen. This interpretation draws isomorphisms with the exclusionary zoning literature, e.g. \cite{keepingpeopleout}. Lastly, the case in which $\epsilon(i) = 0$ corresponds to the supply side in \cite{kulka}. 

\paragraph*{Consumer's Problem}
Households have Cobb-Douglas preferences over a freely traded, homogenous good $g$ (with a normalized price of 1) and housing $A$. Crucially, households differ in units of effective labour $z$ lying in a finite support $Z$ contained in the unit interval. Let $\bar{L}(z)$ be the mass of type $z$ households. Deferring neighborhood choice for a moment, suppose a household of type $z$ has chosen $i$.  Given the city $C(i)$ associated with neighborhood, the household receives a wage $w(C(i)) := w(i)$ per effective unit, and solves 
\begin{equation}\label{utility}
	\max_{A, g} A^{\beta}g^{1-\beta}
\end{equation} 
subject to $A \geq A^{\star}(i)$ and $P(i)A + g \leq w(i)z$. \paragraph*{}
Let $V(P(i), w(i), z, A^{\star}(i)) := V(i, z)$ be the indirect utility associated with \eqref{utility}. If housing is unaffordable at $z$, we set $V(i, z) = 0$. This will guarantee that the household will always prefer an affordable neighborhood to an unaffordable one. Working households do not own land, and thus housing rents are not distributed to them.   

\paragraph*{Neighborhood Choice} I define the \textit{amenity value} $b(i, z)$ of neighborhood $i$ for type $z$ households. As is standard, this amenity enters multiplicatively, so that households recieve a utility of $V(i, z)b(i, z)$. The $b(i, z)$ will play the role of "structural residuals" chosen to rationalize the population distribution and local income distributions after accounting for consumption. Households of all types also draw multiplicative idiosyncratic amenities shocks over neighborhoods. These shocks are distributed multivariate Frechet\footnote{That is, with identical marginal distributions. This is a normalization, as differences in the average amenities shock would be absorbed in the structural residual $b(i, z)$.}, allowing for correlation within cities using a standard copula in the literature \citep{propamp}. Assuming there are a large number of households and shocks are independent across them, the mass of type $z$ households that choose $i$ are
\begin{equation}\label{laboursupply}
	L(i, z) = \bigg[\frac{W(C(i), z)}{\boldsymbol{W}(z)}\bigg]^{\theta}\bigg[\frac{V(i, z)b(i, z)}{W(C(i), z)}\bigg]^{\frac{\theta}{1-\rho}}\bar{L}(z)
\end{equation}
 under the restrictions $\theta > 1$ and $\rho \in [0, 1)$, and where 
 \begin{equation*}
 	W(C(i), z) = \bigg[\sum_{i' \in N(C(i))}\big(V(i', z)b(i', z)\big)^{\frac{\theta}{1-\rho}}\bigg]^{\frac{1-\rho}{\theta}}
 \end{equation*} 
is the expected welfare of a household $z$ who chose a neighborhood in $C(i)$ and 
\begin{equation*}
\boldsymbol{W}(z) = \bigg[\sum_{c \in C} W(c, z)^{\theta}\bigg]^{\frac{1}{\theta}}	
\end{equation*}
is the expected welfare of a type $z$ household before drawing a shock. This is our standard measure of welfare moving forward. $\theta$ governs how responsive migration flows are to changes in neighborhood valuations. $\rho$ governs the responsiveness of migration flows across neighborhoods within a given city relative to across cities. These may differ for salient reasons; for example, moving across cities might require finding both a new home and a new job, while moving within a city may only require finding a new home. If $\rho \to 0$ and $\theta \to \infty$, we are essentially in a open city model with perfect mobility. Likewise, ignoring parameter restrictions, if $\theta \to 0$ but $\frac{\theta}{1-\rho} \to \infty$, we are in a closed city model. Thus, the model doesn't take a stand on the extent to which the world is characterized by one of these polarizing assumptions, which is rare in this literature\footnote{Structural urban models that are confined to a metropolitan area almost always employ a closed city assumption, e.g. \cite{acosta}.}. 

 \paragraph*{Amenities} 
 The income sorting caused by the regulation may have important implications for shaping the spatial pattern of the amenity values $b(i, z)$. Before delving into the microfoundations of these spatial patterns, I specify a relation determining the amenity in my baseline model:
 \begin{equation}\label{endoamen}
 	\log\big[b(i, z)\big] = -\kappa\tau(i) + \Omega\log\bigg[\frac{\sum_{z' \in Z}w(i)z'L(i, z')}{\sum_{z' \in Z}L(i, z')}\bigg] + \epsilon(i, z)
 \end{equation}
 where $\tau(i)$ is the average neighborhood commute time, the second term is income per capita of neighborhood $i$ and $\epsilon(i, z)$ are residuals chosen to match observed population and income distributions. 
 
 \paragraph*{}
 There are at least two main channels that I have emphasized thus far that would proximally give rise to \eqref{endoamen}. Firstly, local income could increase local amenities through variety effects in a Dixit-Stigliz style model \citep{AlmagroDI} \citep{Coutureetal}, while local population could decrease the amenity value through urban congestion effects or from the disutility of density highlighted in \citep{KSC}. When these two forces operate at the same elasticity $\Omega$, amenity values depend only on income per capita\footnote{Separating these two competing forces would otherwise require two instruments, which I am not sure about yet. I could always generalize this later.}. Secondly, local governments could provide a congested public good financed through property taxes \citep{calabresetal}. In that case, income per capita would be replaced with property tax revenue per capita. In a model with Cobb-Douglas preferences, no lot sizes and random heterogeneity in property tax rates, this almost identical to income per capita. With assessment data, I have the luxury of playing around with different specifications as a substitute for model selection. I can also change the spatial scale at which the externality operates, i.e. by grouping tracts into school districts to better reflect the idea of a "local jurisdiction" in the exclusionary zoning literature, or reflecting the spatial diffusion of amenities as in \cite{travelamen} or \cite{suamenities}.
 \paragraph*{}
However, I do not want to limit myself to these interpretations. Instead, I assume $\Omega$ includes all factors that could be \textit{caused} by income per capita. Apart from the above, these may include reduced crime or other types of peer effects. 
 
 
\paragraph*{Production} In each city $c$, production of the numeraire good $g$ takes place at a central business district using the Cobb Douglas technology
\begin{equation}\label{production}
g(c) = \sigma(c)T_{C}(c)^{1- \alpha - \eta}K_{g}(c)^{\eta}Z_{g}(c)^{\alpha}  
\end{equation}
 where $\sigma(c)$ is a city productivity term, $K_{g}(c)$ is the capital employed for the production of $g$, $Z_{g}(c)$ are the effective units of labor employed and $T_{C}(c)$ is the land available for commercial use. As with capital used in the housing sector, capital is supplied with infinite elasticity as rate $r$. This specification follows \cite{hseihmoretti} exactly, and retains the properties that cause the spatial misallocation of labour. The labour demand curve associated with \eqref{production} is
 \begin{equation}\label{labourdemand}
 	Z_{g}(c) = T_{C}(c)\tilde{\sigma}(c)^{\frac{1}{1-\alpha - \eta}}r^{\frac{-\eta}{1-\alpha - \eta}}w(c)^{-\frac{1 - \eta}{1 - \alpha - \eta}}
 \end{equation}
 for $\tilde{\sigma}(c) \propto \sigma(c)$. I adopt this monocentric production model because of data limitations. The model predicts income sorting into commuting because lot sizes may be more stringent far from employment centers, as it appears to be from the observational evidence. In a polycentric model with varying within-city productivity, one would require household level commuting data to parse the selection bias when calibrating this productivity.
 
 \paragraph*{Endowments}Haven't thought much about endowments. This model so far will follow the absentee landlords structure. However, computing realistic welfare effects may require accounting for household tenure. I could also model capital ownership, assuming no international capital flows. These decisions will be made at a later stage. 
 
 
 \paragraph*{Equilibrium} An equilibrium is defined as a set of housing prices $P(i)$, wages $w(c)$, neighborhood allocations $L(i, z)$, amenities $b(i, z)$ and minimum structure requirements $A^{\star}(i)$ such that 
 \begin{enumerate}
 	\item Labour Markets clear: Given indirect utility $V\big(P(i), z, w(c), A^{\star}(i)\big) : = V(i, z)$ solving \eqref{utility}, amenities $b(i, z)$ solving \eqref{endoamen} and labour supply per household type $L(i, z)$ solving \eqref{laboursupply}, we have $\sum_{i \in N(c)}\sum_{z \in Z}zL(i, z)$ equals labour demand \eqref{labourdemand} in every city $c$.
 	
 	\item Housing Markets clear: Given $A^{\star}(i)$ solving \eqref{minstructure} and population $L(i, z)$, the neighborhood demand for housing stock derived from \eqref{utility} equals the neighborhood supply of housing stock \eqref{supplyfn} in every $i$. 
 	
 \end{enumerate}
 
 
\section{Calibration and Identification}
\paragraph*{}
In this section, I sketch a procedure that estimates some parameters and chooses others to rationalize the data as an equilibrium of the model.

\paragraph*{Minimum Lot Sizes} \cite{Song} demonstrates that a structural break detection algorithm on the distribution of local lot sizes works well to identify minimum lot sizes. One issue she faces is how to define "local" -- a geographic region by which to run this algorithm. The issue is that regulation varies widely at the microgeographic level, but one needs a sufficient amount of data both below and above the regulatory minimum. In other words, conditional on identifying a regulatory boundary, one wants to use as much data in that boundary. Unfortunately, data identifying zoning rules from assessments is scarce. Her solution when zoning data are missing is a clustering algorithm on census blocks, grouping them by similar distributions of lot sizes and observable land use. I also will perform this. 
\paragraph*{}
I extend the algorithm beyond single family homes using the \textit{minimum land per housing unit} statistic rather than the lot size directly. This allows for the treatment of duplexes, triplexes, and fourplexes. Moreover, to assign a minimum land per housing unit to a census tract, some aggregation procedure is needed. This is because one may observe regions in which single family homes are in close proximity to dense, multi-story structures by which unit density restrictions are hard to measure. I will only consider assigning minimum lot sizes to locations that have a fraction of housing units in large apartments or condominiums below some threshold. I may let the data choose this threshold by targeting the observed number of housing units in each tract. 


\paragraph*{Housing production functions} Assessment data allow for the construction of high quality measures of housing prices. Following \cite{BSH}, I construct prices derived from a log-linear hedonic regression of housing prices on housing characteristics, with tract-level fixed effects. However, unlike \cite{BSH}, I adopt a general definition of housing stock, and not the observed measure of floor-space. That is, for the unit price of housing stock $P(i)$ from the hedonic regression, and the observed price of a house $\tilde{P}(i)$, I define the housing stock to be $A(i) = \frac{\tilde{P}(i)}{P(i)}$. I adopt this because it is highly plausible that developers respond to large minimum lot sizes by increasing the quality of the inputs into a house, and not necessarily the space it occupies. They note that this distinction yields similar similar estimates of the housing supply elasticity, see their Footnote 4. Lastly, I then construct a total housing stock metric using the average housing stock per housing unit and the number of housing units from the 2010 census. 
\paragraph*{}
With elasticities $\epsilon(i)$\footnote{\cite{BSH} note in their model that the housing supply elasticities are endogenous. In the presence of fixed costs of development, they respond to local prices through parcel selection. I ignore this mechanism for now and instead consider a (log) first order approximation to the housing supply function.}, prices $P(i)$ and housing stock $A(i)$ from the data, the productivity shifter $\lambda(i)r$ is uniquely identified by Equation \eqref{supplyfn}. Moreover, the model-implied minimum housing stock is uniquely identified from \eqref{minstructure}. I will use the observed left tail of the housing quality distribution in a given tract, along with the detected lot sizes to check if these are highly correlated with the model-implied counterpart. 


\paragraph*{City productivity and wages} My model features income sorting into MSAs, leading to a selection problem when trying to identify city productivity. I identify the city wage using a Mincer-type regression with MSA-specific fixed effects, partialing out all household-level observed characteristics that would be correlated with household level wages. This procedure is similarly done in \cite{hseihmoretti}, despite their model not having this feature. I choose city productivity $\sigma(c)$ to uniquely solve \eqref{labourdemand} at the observed equilibrium level of effective units of labour\footnote{This is conditional on choosing parameters for the production function, which I take from the literature.}. This is constructed via the tract-level income distributions in the ACS after adjusting for the city fixed effect. 

\paragraph*{Local consumption indices} Given some value of $\beta$, as well as wages $w(c)$, prices $P(i)$, and minimum housing stock $A^{\star}(i)$ from the above procedures, the consumption indices $V(i, z)$ are uniquely identified from the consumers problem \eqref{utility}. 


\paragraph*{Migration elasticities and endogenous amenities} Fix values of $\theta$ and $\rho$, which can be taken from the literature. For example, \cite{morettihornbeck} estimate $\theta \approx \frac{10}{3}$ using long run time differences and good instruments. Implied values of $\rho$ are estimated in closed city models of commuting, such as \cite{herzog2022} or \cite{severen2021}\footnote{Unforunately, \cite{severen2021} estimates a within-city location choice elasticity \textit{smaller} than the implied cross-city migration elasticity in \cite{morettihornbeck}. This cannot be rationalized under the imposed parameter restrictions in the model.}. Taking the estimated value $\frac{\theta}{1-\rho} = 7.21$ from \cite{herzog2022} and $\theta$ from \cite{morettihornbeck} implies $\rho = 0.5377$. I may estimate these later under some similar identification strategy if people have issue with this. Lastly, average commute times by tract are observed in the ACS, and I take the semi-elasticity of commuting cost with respect to commuting time $\kappa = 0.01$ from \cite{berlinwall}. 
\paragraph*{}
Next, I choose the level of amenities $b(i, z)$ to rationalize the observed population and income distributions given $V(i, z)$, $\theta$ and $\rho$ by solving \eqref{laboursupply}. These local distributions have a higher degree of censoring in the ACS than what I might want. I may choose to use a more granular support of $z$ and only target the more aggregated local distributions.

\paragraph*{}
 What remains is to estimate the elasticity of amenities with respect to income per capita, $\Omega$. Transform the endogenous amenities equation \eqref{endoamen} to yield the equation 
 \begin{equation}
 	\log\big[\tilde{b}(i, z)\big] = \Omega\log\bigg[\frac{\sum_{z' \in Z}w(i)z'L(i, z')}{\sum_{z' \in Z}L(i, z')}\bigg] + \epsilon(i, z)
 \end{equation}
 where $\tilde{b}(i, z) := b(i, z)e^{-\kappa\tau(i)}$ are commuting-cost adjusted amenities.  $\Omega$ can be identified through this implied regression provided the exclusion restriction holds. However, my model provides a micro-foundation \textit{against} this exclusion restriction. Income per capita is simultaneously determined in the model; an exogenous increase in unobserved amenities $\epsilon(i, z)$ causes increases in housing prices through standard channels, whereby increasing the stringency of lot size regulation through Equation \eqref{minstructure}, and thus more income sorting. The relationship between income sorting and house prices is also present in models with non-homothetic preferences over housing, as in \cite{superstarcities} and \cite{LeeandLin}. Moreover, a similar simultaneity bias arises in models with endogenous amenities and heterogenous valuations of those amenities, such as \cite{diamond2016}. 
 \paragraph{}
 Apart from that, there may be exogenous causal factors contained in the error terms. By definition, these factors would not respond to local changes in average income, and thus confound the effects of falling regulation. Plausible forces include the "anchoring" effects of time-invariant natural amenities as in \cite{LeeandLin}, which I will control for using their tract-level dataset. 
 
 \paragraph*{}
 I propose an instrument to identify $\Omega$: tract level terrain slopes. These may obviously be correlated with natural amenities: more slopes imply more opportunity for nice views, which high income households may value more. I hope to control for a bulk of other natural amenities to limit the bias. I also include MSA FE's to limit the comparison of topographically heterogenous regions\footnote{I could also think about estimating the regression in changes, provided slopes are correlated with extranormal housing price growth and thus income sorting. }. It turns out that this instrument is \textit{extremely} relevant using 2008-2012 ACS data merged with topographical data from \cite{LeeandLin}\footnote{The first stage includes controls for distances to rivers, lakes, oceans, precipitation, max july and min january temperature, as well as MSA fixed effects and clustering at the MSA level. I also include a set of control variables that captures the idea that these locations provide nice views -- the average slope divided by the distance to rivers, lakes and oceans. This does not change the relevance of the instrument. The Cragg-Donald F statistic is 1500 and the Kleibergen-Paap F statistic is 56.}. This is perhaps not surprising; this was an observation made in \cite{saiz2010}. Moveover, this regression avoids the critique made in \cite{davidoffcritique} by directly modeling the relationship that invalidates the use of this instrument in other contexts. 
 


\newpage
\scriptsize
\bibliography{references.bib}
\end{document}
