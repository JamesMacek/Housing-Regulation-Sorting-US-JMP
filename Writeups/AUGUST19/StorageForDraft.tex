\begin{comment}
	UPDATE IDENTIFICATION STRATEGY WITH SOMETHING A BIT BETTER!
	
	$\Omega$ can be identified through this implied regression provided the exclusion restriction holds. However, my model provides a micro-foundation \textit{against} this exclusion restriction. Income per capita is simultaneously determined in the model; an exogenous increase in unobserved amenities $\epsilon(i, z)$ causes increases in housing prices through standard channels, whereby increasing the stringency of lot size regulation through Equation \eqref{minstructure}, and thus more income sorting. The relationship between income sorting and house prices is also present in models with non-homothetic preferences over housing, as in \cite{superstarcities} and \cite{LeeandLin}. Moreover, a similar simultaneity bias arises in models with endogenous amenities and heterogenous valuations of those amenities, such as \cite{diamond2016}. 
	\paragraph{}
	Apart from that, there may be exogenous causal factors contained in the error terms. By definition, these factors would not respond to local changes in average income, and thus confound the effects of falling regulation. Plausible forces include the "anchoring" effects of time-invariant natural amenities as in \cite{LeeandLin}, which I will control for using their tract-level dataset. 
	
	\paragraph*{}
	I propose an instrument to identify $\Omega$: tract level terrain slopes. These may obviously be correlated with natural amenities: more slopes imply more opportunity for nice views, which high income households may value more. I hope to control for a bulk of other natural amenities to limit the bias. I also include MSA FE's to limit the comparison of topographically heterogenous regions\footnote{I could also think about estimating the regression in changes, provided slopes are correlated with extranormal housing price growth and thus income sorting. }. It turns out that this instrument is \textit{extremely} relevant using 2008-2012 ACS data merged with topographical data from \cite{LeeandLin}\footnote{The first stage includes controls for distances to rivers, lakes, oceans, precipitation, max july and min january temperature, as well as MSA fixed effects and clustering at the MSA level. I also include a set of control variables that captures the idea that these locations provide nice views -- the average slope divided by the distance to rivers, lakes and oceans. This does not change the relevance of the instrument. The Cragg-Donald F statistic is 1500 and the Kleibergen-Paap F statistic is 56.}. This is perhaps not surprising; this was an observation made in \cite{saiz2010}. Moveover, this regression avoids the critique made in \cite{davidoffcritique} by directly modeling the relationship that invalidates the use of this instrument in other contexts. 
\end{comment}

\begin{comment}
	\section{Calibration and Identification}
	\paragraph*{}
	In this section, I sketch a procedure that estimates some parameters and chooses others to rationalize the data as an equilibrium of the model.
	
	\paragraph*{Minimum Lot Sizes} \cite{Song} demonstrates that a structural break detection algorithm on the distribution of local lot sizes works well to identify minimum lot sizes. One issue she faces is how to define "local" -- a geographic region by which to run this algorithm. The issue is that regulation varies widely at the microgeographic level, but one needs a sufficient amount of data both below and above the regulatory minimum. In other words, conditional on identifying a regulatory boundary, one wants to use as much data in that boundary. Unfortunately, data identifying zoning rules from assessments is scarce. Her solution when zoning data are missing is a clustering algorithm on census block groups, grouping them by similar distributions of lot sizes and observable land use. I also will perform this. 
	\paragraph*{}
	I extend the algorithm beyond single family homes using the \textit{minimum land per housing unit} statistic rather than the lot size directly. This allows for the treatment of duplexes, triplexes, and fourplexes. Moreover, to assign a minimum land per housing unit to a block group, some aggregation procedure is needed. This is because one may observe regions in which single family homes are in close proximity to dense, multi-story structures by which unit density restrictions are hard to measure. I will only consider assigning minimum lot sizes to locations that have a fraction of high density structures below some threshold.
	
	\paragraph*{Housing production functions} Assessment data allow for the construction of high quality measures of housing prices. Following \cite{BSH}, I construct prices derived from a log-linear hedonic regression of housing prices on housing characteristics, with tract-level fixed effects. However, unlike \cite{BSH}, I adopt a general definition of housing stock, and not the observed measure of floor-space. That is, for the unit price of housing stock $P(i)$ from the hedonic regression, and the observed price of a house $\tilde{P}(i)$, I define the housing stock to be $A(i) = \frac{\tilde{P}(i)}{P(i)}$. I adopt this because it is highly plausible that developers respond to large minimum lot sizes by increasing the quality of the inputs into a house, and not necessarily the space it occupies. They note that this distinction yields similar similar estimates of the housing supply elasticity, see their Footnote 4. Lastly, I then construct a total housing stock metric using the average housing stock per housing unit and the number of housing units from the 2010 census. 
	\paragraph*{}
	With elasticities $\epsilon(i)$\footnote{\cite{BSH} note in their model that the housing supply elasticities are endogenous. In the presence of fixed costs of development, they respond to local prices through parcel selection. I ignore this mechanism for now and instead consider a (log) first order approximation to the housing supply function.}, prices $P(i)$ and housing stock $A(i)$ from the data, the productivity shifter $\lambda(i)r$ is uniquely identified by Equation \eqref{supplyfn}. Moreover, the model-implied minimum housing stock is uniquely identified from \eqref{minstructure}. I will use the observed left tail of the housing quality distribution in a given tract, along with the detected lot sizes to check if these are highly correlated with the model-implied counterpart. 
	
	
	\paragraph*{City productivity and wages} My model features income sorting into MSAs, leading to a selection problem when trying to identify city productivity. I identify the city wage using a Mincer-type regression with MSA-specific fixed effects, partialing out all household-level observed characteristics that would be correlated with household level wages\footnote{This procedure follows \cite{whyurbineq} closely, and will appear in future drafts.}. This is constructed via the tract-level income distributions in the ACS after adjusting for the city fixed effect. 
	
	\paragraph*{Local consumption indices} Given some value of $\beta$, as well as wages $w(c)$, prices $P(i)$, and minimum housing stock $A^{\star}(i)$ from the above procedures, the consumption indices $V(i, z)$ are uniquely identified from the consumers problem \eqref{utility}. 
	
	
	\paragraph*{Migration elasticities and endogenous amenities} Fix values of $\theta$ and $\rho$, which can be taken from the literature. For example, \cite{morettihornbeck} estimate $\theta \approx \frac{10}{3}$ using long run time differences and good instruments. Implied values of $\rho$ are estimated in closed city models of commuting, such as \cite{herzog2022} or \cite{severen2021}\footnote{Unforunately, \cite{severen2021} estimates a within-city location choice elasticity \textit{smaller} than the implied cross-city migration elasticity in \cite{morettihornbeck}. This cannot be rationalized under the imposed parameter restrictions in the model.}. Taking the estimated value $\frac{\theta}{1-\rho} = 7.21$ from \cite{herzog2022} and $\theta$ from \cite{morettihornbeck} implies $\rho = 0.5377$. I may estimate these later under some similar identification strategy if people have issue with this. Lastly, average commute times by tract are observed in the ACS, and I take the semi-elasticity of commuting cost with respect to commuting time $\kappa = 0.01$ from \cite{berlinwall}. 
	\paragraph*{}
	Next, I choose the level of amenities $b(i, z)$ to rationalize the observed population and income distributions given $V(i, z)$, $\theta$ and $\rho$ by solving \eqref{laboursupply}. These local distributions have a higher degree of censoring in the ACS than what I might want. I may choose to use a more granular support of $z$ and only target the more aggregated local distributions.
	
	\paragraph*{}
	What remains is to estimate the elasticity of amenities with respect to income per capita, $\Omega$. Transform the endogenous amenities equation \eqref{endoamen} to yield the equation 
	\begin{equation}
		\log\big[\tilde{b}(i, z)\big] = \Omega\log\bigg[\frac{\sum_{z' \in Z}w(i)z'L(i, z')}{\sum_{z' \in Z}L(i, z')}\bigg] + \epsilon(i, z)
	\end{equation}
	where $\tilde{b}(i, z) := b(i, z)e^{-\kappa\tau(i)}$ are commuting-cost adjusted amenities. 
%\end{comment} 



\section{Literature}
\paragraph*{}
Given the spatial scale of the model, I see this paper as firstly improving upon the literature that quantifies the macroeconomic impacts of housing regulation. These papers generally employ variations of the Rosen-Roback framework, consider homogenous households, and derive welfare implications from calibrated models. The two most relevant to my work are \cite{hseihmoretti} and \cite{durantonpugaurbgrowth}. Firstly, \cite{hseihmoretti} show that growth in the US was slowed because this growth was concentrated in cities with inelastic housing supply. Importantly, they also measure the welfare impacts of regulation through their effect on housing supply elasticities, using estimates from \cite{saiz2010}. On the other hand, \cite{durantonpugaurbgrowth} consider a model (FINISH). 
\paragraph*{}
\cite*{hop} use model-based residuals to measure land use regulation over time, and perform similar counterfactuals. \cite{ganongshoag} link housing regulation to declining spatial convergence in US income. \cite{parkho} and \cite{bunten} construct aggregate welfare estimates of deregulation in settings where they are determined in a political economy equilibrium, and find attenuated gains\footnote{These estimates are also not directly comparable to \cite{hseihmoretti} because they are built under models with local increasing returns to scale. That is, the misallocation mechanism of \cite{hseihmoretti} is not present in them.}.  The former also considers national policies and how local regulation might respond to them. On a smaller spatial scale, \cite{martynov} and \cite{acosta} measure the impacts of housing stock density restrictions in a model where commuting plays a first order role. 

\paragraph*{}
I build on all these papers by focusing on the income sorting mechanism both within and across cities, and consider that household movement within those cities to avoid stringent lot size regulation. In doing so, I draw ideas from other literatures under the umbrella of urban and public economics. I review them in sequence:

\paragraph*{Housing regulation} This paper adds to the historic and fast growing literature that studies housing regulation, as recently reviewed in \cite{gyourkomolloy}. In the context of minimum lot sizes, recent quasi-experimental evidence from \cite{Song} and \cite{kulka} show extreme sorting on income and race into stringent municipalities. \cite{KSC} and \cite{zabel} find significant effects of similar regulations on local prices. These papers provide a strong motivation for thinking about these channels in a general equilibrium context. Importantly, the former papers' models do not incorporate endogenous amenities\footnote{\cite{Song} directly models preferences over minimum lot sizes in a discrete choice model. This may be thought of as a stand-in for endogenous amenities, though this remains unclear.}, nor do they model heterogeneous local technologies that are important for assessing the aggregate welfare consequences of deregulation.
\paragraph*{}
Beyond these, the literature on lot sizes and housing unit density restrictions is surprisingly scant relative to research on housing \textit{stock} density restrictions. Theoretical treatments include \cite{bbheight} and \cite{mills2005}, who show these restrictions cause urban sprawl. On the empirical side, \cite{BruecknerFuGu} and \cite{bruecknersingh} use a simple model to measure the stringency of FAR restrictions in China and the US. \cite{anagoletal2021} find regressive gains to loosening FAR restrictions in Sao Paulo. These restrictions are sometimes thought to be isomorphic to minimum lot sizes, but I reiterate that they are not --  a distinction made in an old literature \citep{griesonwhite}.


\paragraph*{Urban spatial sorting} A growing literature in urban economics is seeking to understand the causes and consequences of the spatial sorting of different groups, both within and across cities. Usually, these papers stress that sorting in response to changing exogenous fundamentals are exacerbated by endogenous forces, such as changing amenities. \cite{diamond2016} show that the rapidly evolving skill intensity in large cities was largely driven through amenities valued by the high skilled. \cite{couturehandbury} and \cite{su2021} show that non-traded amenities and the rising value of time have contributed to the changing skill intensity in downtowns, respectively. \cite{bshartley2020} show a similar result for recently observed racial sorting. \cite{Coutureetal} show that the evolving national income distribution has contributed to the recent gentrification in US downtowns.

\paragraph*{}
Unlike these papers, regulated lot sizes are the anchor that cause neighborhood income sorting in my model. However, I want to stress that it cannot explain these recent gentrification patterns, and instead speaks to an older literature on urban decay; including one that explains negative income sorting into downtowns \citep{parispoor} \citep{ccpoortransport}. Moreover, my model cannot explain how income sorting is shaped by durable housing and the spatiotemporal patterns of filtering \citep{Gentrificationcycles}. 

\paragraph*{}
There may also be a link with the urban inequality literature, though I have yet to flush that out in a model \citep{ineqcitysize}, \citep{spatialsorting}, \citep{FogliGuerrieri}. Within-city heterogeneity in lot sizes may play a role.

\paragraph*{Exclusionary zoning}
This paper is also closely linked to a literature which studies the role that zoning policy plays in the presence of Tiebout sorting and ad valorem property taxes. In general, these papers consider models where households vote over the minimum amount of housing services households must purchase to live in a jurisdiction. In equilibrium, rich households exclude the poor to bolster their fiscal surplus with respect to a local public good. \cite{calabresetal} and \cite{keepingpeopleout} consider a model where zoning policy is collectively chosen before and after neighborhood choice, respectively. They both generally find regressive welfare effects, but find substantial efficiency gains. In contrast, \cite{ineffTiebout} calibrate a model where fiscal decentralization fails to materialize into efficiency gains. A similar conclusion is reached by \cite{barcoate} in a dynamic framework. In my model, the externality that gives rise to urban spatial sorting can be interpreted in the context of these papers. However, I do not endogenize the choice of local zoning policy, and instead consider it as a lever than can be freely changed. 

