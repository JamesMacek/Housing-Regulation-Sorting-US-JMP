\subsection{The sources of the gains to deregulation}
\paragraph*{}
Start intro-- show why we'd care about each of the three welfare channels to think about a decomposition. 


%\paragraph*{} Regulation may decrease the consumption value of a neighborhood $V(i, z, \tau)$ because it constrains the choice of housing consumption. Moreover, regulation keeps housing prices $P(i)$ high in equilibrium by forcing low income households to consume more housing than they would under a regime with no regulation\footnote{Prices need to adjust upward to clear housing markets if households are consuming more housing under the regulation.}. If mobility across neighborhoods is restricted, this means that regulation cannot make any renter worse off. Let $\Delta\log[V(i, z, \tau)]_{N}$ be the resulting change in consumption value after some change in regulation $l(i)$ while restricting mobility across neighborhoods or cities. 

%\paragraph*{} Allowing for within-city labour mobility means that neighborhood amenities can adjust to changes in the local income composition of residents. Moreover, within-city movement changes local housing demand and thus equilibrium housing prices $P(i)$. In an equilibrium where mobility is free within cities but completely restricted across cities, let the resulting change in neighborhood $i$ value be $\Delta\log[V(i, z, \tau)]_{C} + \Delta\log[b(i, z)]_{C}$ in response to some change in regulation. If $\mathbb{I}(i)$ is income per capita in $i$, $\Delta\log[b(i, z)]_{C} = \Omega\Delta\log[\mathbb{I}(i)]_{C}$ by Equation \eqref{endoamen}. Allowing households to move both within and across cities induces welfare gains or losses under the exact same set of channels, but the relative strength of these channels may differ. Let $\Delta\log[V(i, z, \tau)]_{GE} + \Omega\Delta\log[\mathbb{I}(i)]_{GE}$ be the resulting change in neighborhood $i$ valuation under a full general equilibrium response to deregulation.


%\paragraph*{}The change in aggregate renter welfare $\boldsymbol{W}(z, \tau)$ from Equation \eqref{laboursupply} can be computed and compared across each of the three scenarios $ s \in \{N, C, GE\}$, as summarized by Proposition \ref{Prop:WelfareDecomp}.

\begin{Proposition}\label{Prop:WelfareDecomp}
	For each $s \in \{N, C, GE\}$, \\
	\begin{equation*}
		\Delta \log[\boldsymbol{W}(z, \tau)]_{s} = \sum_{i \in \cup_{c \in C}N(c)}\frac{L(i, z, \tau)}{\bar{L}(z, \tau)}\bigg(\Delta\log[V(i, z, \tau)]_{s} + \Omega\Delta\log[\mathbb{I}(i)]_{s}\bigg)
	\end{equation*}
	where $\{L(i, z, t)\}_{i \in \cup_{c\in C}N(c)}$ is an equilibrium allocation of $(z, \tau)$ households before deregulation. 
\end{Proposition}

Proposition \ref{Prop:WelfareDecomp} says that the change in aggregate welfare after deregulation is related to initial population weighted changes in consumption values and per capita incomes. Building on this, I define the change in \textit{social welfare} using weights proportional to the mass of $(z, \tau)$ households:

\begin{equation*}
	\Delta \log[\boldsymbol{W}]_{s} := \sum_{z\in Z}\int_{0}^{\infty}\frac{\bar{L}(z, \tau)}{\bar{L}}\Delta \log[\boldsymbol{W}(z, \tau)]_{s}d\tau
\end{equation*}

Proposition \ref{Prop:WelfareDecomp} gives some key insight into what separates the sources of aggregate welfare gains in a model where across city mobility is restricted ($s = C$) and one where no mobility is restricted ($s = GE$). In the main counterfactual, I find essentially no difference between $\Delta \log[\boldsymbol{W}]_{C}$ and $\Delta \log[\boldsymbol{W}]_{GE}$. Ideally, by Proposition \ref{Prop:WelfareDecomp}, welfare gains are maximized by concentrating consumption and amenity gains in locations with the highest initial populations. These tend to be the most expensive and most dense cities. However, if regulation causes income sorting, we'd observe large negative changes in consumption and amenity value in these cities for high income households, and vice versa for low income households. These two opposing forces offset completely. 
